\documentclass{article}

\title{Software Reliability -- Coursework 1 \\ Using ESC/Java 2 and Daikon}
\author{Jack Bracewell, Milan Misak, Craig Ellis}
\date{}

\usepackage{float}

\begin{document}

\maketitle

\section{Introduction}

Not sure we need an introduction

\section{ESC/Java}

\subsection{Eliminating ``null'' warnings}

One of the interesting things about this bit is the order of comment annotations.

initially we tried:

\begin{verbatim}
  /* invariant \nonnullelements(seatReservations)
  /*@ non_null */
  private final Customer[][] seatReservations;
\end{verbatim}

This actually created 2 warnings regarding assigning null to fields of seatReservations. It seems that the
\verb|non_null| comment annotation overwrote the invariant annotation. Switching the order to:
\begin{verbatim}
  /*@ non_null */
  /* invariant \nonnullelements(seatReservations)
\end{verbatim}
fixed removed the warnings.

\subsection{Eliminating ``negative length'' warnings}

\subsection{Eliminating ``negative array index'' warnings}

These cases were all very similar, and were related to the two fields (row, number) of the Seat class. To solve them, we first gave the getters post-conditions that ensured their values were in an acceptable range (Seat.MIN\_ROW, MAX\_ROW; Seat.MIN\_NUMBER, MAX\_NUMBER). This required two other changes - we made gave invariants to both row and number, and preconditions to their setters. We also made the indexToNumber() and indexToRow() functions into helpers.

\subsection{Eliminating ``index too large'' warnings}

\subsection{Eliminating ``array element subtyping'' warnings}

\subsection{Eliminating ``violation of modifies clause'' warnings}

\section{Daikon}

\end{document}
